%% vim: set spell spelllang=en:
\documentclass[runningheads]{llncs}

\usepackage[latin1]{inputenc}
\usepackage{graphicx}
\usepackage{xspace}
\usepackage{amsmath}
\usepackage{amssymb}
\usepackage{stmaryrd}
\usepackage{ifpdf}
\usepackage[all]{xy}

\ifpdf
%    \usepackage{hyperref}
   \DeclareGraphicsRule{*}{mps}{*}{}
 \else
\fi

\pagenumbering{arabic}

\newcommand{\scala}{\textsc{Scala}}
\newcommand{\pizza}{\textsc{Pizza}}
\newcommand{\jmatch}{\textsc{JMatch}}
\newcommand{\jbossrules}{\textsc{JBoss Rules}}
\newcommand{\jess}{\textsc{Jess}}
\newcommand{\jrules}{\textsc{JRules}}
\newcommand{\stratego}{\textsc{Stratego}}
\newcommand{\jjtraveler}{{JJTraveler}}
\newcommand{\maude}{\textsc{Maude}\xspace}
\newcommand{\asfsdf}{{ASF+SDF}\xspace}

\newcommand{\elan}    {\textsf{ELAN}\xspace}
\newcommand{\tom}{\textsc{Tom}}
\newcommand{\gom}{\textsc{Gom}}
\newcommand{\java}{\textsc{Java}}
\newcommand{\C}{\textsf{C}}
\newcommand{\eclipse}{\textsc{Eclipse}}
\newcommand{\ocaml}{\textsc{OCaml}}
\newcommand{\ml}{\textsc{ML}}
\newcommand{\haskell}{\textsc{Haskell}}
\newcommand{\fsharp}{\textsf{F \#}}
\newcommand{\lex}[1]{{\textrm{\textbf{#1}}}}

\newcommand{\isdef}{\mathrel{\mbox{\small$\stackrel{\mbox{\tiny$\triangle$}}{=}$}}}
\def\vs{\vspace{-1.5mm}}
\newcommand{\Mu}{{\ensuremath{\mu}}}

\newcommand{\ie}{\textit{i.e.}}
\newcommand{\wrt}{\textit{wrt.}}
\newcommand{\etc}{\textit{etc.}}

% code samples
\RequirePackage{listings}
\lstset{basicstyle={\ttfamily},
        keywordstyle={\rmfamily\bfseries},
        columns=flexible}
\lstdefinelanguage{gom}{
  alsoletter={\%},
  morekeywords={\%match,module,imports,abstract,
                syntax,make,make_insert,realMake},
  sensitive=true,
  morecomment=[l]{//},
  morecomment=[s]{/*}{*/},
  morestring=[b]",
}
\lstnewenvironment{gomcode}[1][]%
{\lstset{language={gom},frame=tb,#1}}
{}
\lstnewenvironment{tomcode}[1][]%
{\lstset{language={java},
  alsoletter={\%},
  morekeywords={\%typeterm,\%op,\%oplist,\%strategy,\%match,
    is_fsym,get_slot,get_head,get_tail,is_empty,implement,equals,
    make,make_empty,make_insert,realMake,\%gom,visit,
    module,imports,abstract,syntax},
  frame=tb,#1}}
{}
\lstnewenvironment{objcode}[1][]%
{\lstset{
  morekeywords={op,strat},
  frame=tb,#1}}
{}
\lstnewenvironment{asfsdfcode}[1][]%
{\lstset{
  morekeywords={module,imports,exports,context-free,syntax,equations},
  frame=tb,#1}}
{}
\lstnewenvironment{javacode}[1][]%
{\lstset{language={java},frame=tb,#1}}
{}
\lstnewenvironment{coqcode}[1][]%
{\lstset{
  morekeywords={Ltac,Inductive,Fixpoint,Theorem,Lemma,match,with,end},
  xleftmargin=1em,#1}}
{}
\lstnewenvironment{elancode}[1][]%
{\lstset{
  morekeywords={where,end},
  frame=tb,#1}}
{}


\title{Strategic rewriting on existing ASTs}

\author{Emilie Balland \and Pierre-Etienne Moreau \and Nicolae Vintila}

\institute{INRIA \& LORIA,\\
  BP 101, 54602 Villers-l{\`e}s-Nancy Cedex France\\
\email{\{Emilie.Balland,Pierre-Etienne.Moreau\}@loria.fr,nvintila@yahoo.com}}

\begin{document}

\maketitle

\begin{abstract}

\end{abstract}

\section{Introduction}

\section{Tom: strategic rewriting piggybacked on Java}
	
The {\tom} language embed strategic rewriting statements in  mainstream
languages like {\java} or {\C}.  For example, the matching statement is
introduced by the \lex{\%match} token and can be used as any {\java} code
block.  This construction is an extension of the conventional construction
\texttt{switch/case}, whose main difference is that the discrimination is
based on algebraic \emph{terms} rather than atomic values as integers or
characters.

A basic example of {\tom} programs is the definition of the addition on Peano
integers. Integers are represented by algebraic terms based on the
\texttt{zero()} constant and the \emph{successor}, \texttt{suc(x)},  which
takes a Peano integer as argument.  The addition is defined by the following
{\java} method:

\medskip
\begin{tomcode}
public int plus(int t1, int t2) {
  %match(t1, t2) {
    x,zero()  -> { return `x; }
    x,suc(y)  -> { return `suc(plus(x,y)); }
  }
}

public void run() {
    System.out.println("plus(1,2) = " + plus(1,2));
} 
\end{tomcode}

In this example, given two terms  $t_1$ and $t_2$ representing Peano integers,
the evaluation of the \texttt{plus} function is implemented by matching: $x$
matches $t_1$ and the $zero()$ and $suc(y)$ patterns match eventually $t_2$.
When the $zero()$ pattern matches $t_2$, the result is $x$, which
is intanciated by $t_1$. When the $suc(y)$ pattern matches $t_2$, it means
that the $t_2$ term is rooted by the $suc$ symbol; the subterm $y$ is added
to $x$, and the successor of this term is returned. The backquote construction
\texttt{`} enables the construction of new algebraic terms and can reuse
variables instanciated by matching.

Note that the matching statement proposed by {\tom} is more expressive than
ones proposed by functional languages. For example,  it is possible to use
equational matching or negative matching. Moreover, the right hand side of the
match statement are not restricted to terms but can be any instruction from the
host language. As for the \texttt{switch/case} statement, if the match
succeeds, the instructions are executed and if the control flow of the program
is not interrupted, the following patterns are tried. The \texttt{plus} method
of the~\ref{ex:tom:match} example uses the {\java} \texttt{return} statement in
order to interrupt the control flow when a pattern succeeds. Even if this
\texttt{plus} method is defined by matching, it can be used as any other
{\java} method in the surrounding {\java} code. 

When using rewriting as a programming or modeling paradigm, it is common to
consider rewrite systems that are non-confluent or non-terminating. To be able
to use them, it is necessary to exercise some control over the application of
the rules. In {\tom}, a solution would be to use {\java} to express the control
needed. While this solution provides a huge flexibility, its lack of
abstraction renders difficult the reasoning about such transformations.
Strategies such as \emph{bottom-up}, \emph{top-down} or \emph{leftmost-innermost}
are higher-order features that describe how rewrite rules should be applied.
We have developed a flexible and expressive strategy language inspired by
{\elan}, {\stratego}, and {\jjtraveler}~\cite{visser-oopsla01} where high-level
strategies are defined by combining low-level primitives. For example, the
\emph{top-down} strategy is recursively defined by
\texttt{TopDown(s)}~${\isdef}$~\texttt{Sequence(s,All(TopDown(s)))}.

An elementary strategy corresponds to a minimal transformation. It could be
\emph{Identity} (does nothing), \emph{Fail} (always fails), or a set of
\emph{rewrite rules} (performs an elementary rewrite step only at the root
position).  In our system, strategies are type-preserving and have a default
behavior (introduced by the keyword \texttt{extends}) that can be either
\texttt{Identity} or \texttt{Fail}:

\begin{tomcode}
  %strategy R() extends Fail() {
    visit Nat {
      zero()      -> { return `suc(zero()); }
      suc(suc(x)) -> { return `x; }
    }
  }
\end{tomcode}

When a strategy is applied to a term~$t$, as in a \texttt{\%match}, a rule is
fired if a pattern matches. Otherwise, the default strategy is applied.
For example, applying the strategy \texttt{R()} to the term
\texttt{suc(suc(zero()))} will produce the term \texttt{zero()} thanks to the
second rule. The application to \texttt{suc(suc(suc(zero())))} fails since no
pattern matches at root position.

More control is obtained by combining elementary strategies with \emph{basic
combinators} such as \texttt{Sequence}, \texttt{Choice},
\texttt{All}, \texttt{One} as presented
in~\cite{BKK98,visser-icfp98}.

By denoting \texttt{s[t]} the application of the strategy~\texttt{s} to the
term~\texttt{t}, the \emph{basic combinators} are defined as follows:

\begin{small}
\begin{tabular}{lll}

\texttt{Sequence(s1,s2)[t]} & $\rightarrow$ & \texttt{s2[t']} if \texttt{s1[t]}
$\rightarrow$ \texttt{t'}\\
&& failure if \texttt{s1[t]} fails\\

\texttt{Choice(s1,s2)[t]} & $\rightarrow$ & \texttt{t'} if \texttt{s1[t]}
$\rightarrow$ \texttt{t'}\\
&& \texttt{s2[t']} if \texttt{s1[t]} fails\\

\texttt{All(s)[f(t1,\ldots,tn)]} & $\rightarrow$ & \texttt{f(t1',\ldots,tn')}
if \texttt{s[t1]} $\rightarrow$ \texttt{t1'},\ldots, \texttt{s[tn]}
$\rightarrow$ \texttt{tn'}\\
&& failure if there exists \texttt{i} such that \texttt{s[ti]} fails\\

\texttt{One(s)[f(t1,\ldots,tn)]} & $\rightarrow$ &
\texttt{f(t1,\ldots,ti',\ldots,tn)} if \texttt{s[ti]}  $\rightarrow$
\texttt{ti'}\\
&& failure if for all \texttt{i}, \texttt{s[ti]} fails\\
\end{tabular}
\end{small}

An example of composed strategy is
\texttt{Try(s)}~${\isdef}$~\texttt{`Choice(s,Identity())},
which applies~\texttt{s} if it can, and performs the \textit{Identity} otherwise.
To define strategies such as \emph{repeat}, \emph{bottom-up}, \emph{top-down},
{\etc} recursive definitions are needed. For example, to repeat the application
of a strategy~\texttt{s} until it fails, we consider the strategy
\texttt{Repeat(s)}~${\isdef}$~\texttt{Choice(Sequence(s,Repeat(s)),}
\texttt{Identity())}.  In {\tom}, we use the recursion operator~$\Mu$
(comparable to \texttt{rec} in {\ocaml}) to have stand-alone definitions:
$\Mu$\texttt{x.Choice(Sequence(s,x),Identity())}.

The \texttt{All} and \texttt{One} combinators are used to define tree
traversals. For example, we have
\texttt{TopDown(s)}~${\isdef}$~$\Mu$\texttt{x.Sequence(s,All(x))}:
the strategy
 \texttt{s} is first applied on top of the considered term, then
the strategy \texttt{TopDown(s)} is recursively called on all immediate
subterms of the term.

Strategy expressions can have any kind of parameters. It is useful to have a
{\java} \texttt{Collection} as parameter in order to collect information.  For
example, let us consider the following strategy which collects the direct
subterms of an~$f$.  This program creates a hash-set, and a strategy applied to
\texttt{f(f(a()))} collects all the subterms which are under an~\texttt{f}:
{\ie} $\{\texttt{a()}, \texttt{f(a())}\}$.

\begin{tomcode}
  %strategy Collect(c:Collection) extends Identity() {
    visit T {
      f(x) -> { c.add(`x); }
    }
  }
  Collection bag = new HashSet();
  `TopDown(Collect(bag)).apply( `f(f(a())) );
\end{tomcode}

\section{Mapping existing ASTs}

In this section, we will present how the {\tom} language can be directly used
on any Java objects using a mechanism of mappings. In particular, we will show
that it can be used to manipulate in a direct way ASTs like Eclipse's JDT.

\subsection{Mapping data-structures}

In the previous example, we can notice that the {\java} \texttt{plus} method
take two parameters of type \texttt{int} while the match statement is specified
on the {\tom} type \texttt{Nat} composed of the \texttt{zero} and \texttt{suc}
constructors. This shows an important specificity of the {\tom} language : the
matching compilation is implemented independently of the concrete
implementation of the terms. In fact, the match constructs are expressed in
function of algebraic data types and the compiler translates these statements
by manipulation on the concrete data types that represent the algebraic terms.
{\tom} users have to specify the relation (\emph{mapping}) between the {\tom}
algebraic data types and the concrete {\java} types using dedicated {\tom}
syntactic constructs. The \lex{\%typeterm}, \lex{\%op} and \lex{\%oplist}
enable to respectively  describe the implementation of algebraic types,
constructors and list constructors. Now, we will illustrate these constructions
by defining the mapping for Peano integers and the {\java} primitive type
\texttt{int}.

First, the \lex{\%typeterm} construction specifies an algebraic {\tom} type and
its concrete {\java} type. The user has also to declare how testing the
equality between two terms using the concrete representations.

\medskip
\begin{tomcode}[morekeywords={equals}]
%typeterm Nat { 
    implement { int } 
    equals(t1,t2) { t1 == t2 } 
} 
\end{tomcode}

The \texttt{nat} type is declared by the \lex{\%typeterm} construction and is
implemented by the \texttt{int} type. The mapping also specifies that the
equality tests of two \texttt{Nat} terms can be achieved simply by comparison
of their concrete representations (using \texttt{==}). Operators of this kind
are defined using the construction \lex{\%op} which allows to specify both how
\emph{building} and \emph{destroying} (in the sense of decomposing) a term
whose head is the symbol of the declared operator. We can define the
\texttt{zero} and \texttt{suc} constructors as follows:

\begin{tomcode}
%op Nat zero() { 
  is_fsym(i) { i==0 } 
  make() { 0 } 
} 

%op Nat suc(p:Nat) { 
  is_fsym(i) { i>0 } 
  get_slot(p,i) { i-1 } 
  make(i) { i+1 } 
} 
\end{tomcode}

	
The first line of each \lex{\%op} construction defines the signature of the
algebraic operator, and the names of its arguments. The \lex{is\_fsym}
construction used to test whether an object represents a term whose head symbol
is the operator and the \lex{get\_slot} construction can extract the various
sub-terms according to their behalf. Both buildings are the destructive part of
the mapping used to compile the match statement.  The \texttt{make} construction
is used to specify how to construct a term whose head symbol is this operator
and whose subterms are given as parameters.

Variadic operators are defined similarly using the \lex{\%oplist} construction.
The first line specifies the domain and co-domain of the operator, as its name
suggests. Specific operations to lists must be defined, they are used to
compile the filtering list. Mappings specifies how to construct an empty list,
and insert a new element at the top of a list. It also details how
deconstructing a list, by separating the head element and the rest of
the list.

\begin{tomcode}[morekeywords={equals}]
  %typeterm List {
    implement { MyIntList }
    equals(l1,l2)  { l1.equals(l2) }
  }

  %oplist List conc( Nat* ) {
    is_fsym(t)       { t instanceof MyIntList  }
    make_empty()     { new MyIntList() }
    make_insert(e,l) { new MyIntList(e,l) }
    get_head(l)      { l.get(0) }
    get_tail(l)      { l.sublist(1,l.size()) }
    is_empty(l)      { l.isEmpty() }
 }
\end{tomcode}

	
The definition \lex{\%oplist} for the list of \texttt{Nat} elements is
implemented by the class \texttt{MyIntList} which extends
\texttt{ArrayList<integer>} offers functions to construct an empty list and a
list from element of his head and a list corresponding to the tail.  A mapping
should not necessarily be complete. It is sometimes useful to specify only the
destructive part. For example, the {\tom} runtime library provides mappings for
partial standard implementations of the {\java} collection library
(\texttt{java.util} package) which can be used to match any {\java} collection
in a declarative way.

\emph{Give an example with HashMap}

\subsection{Strategic rewriting for existing data-structures}

Introspectors

\section{Application to JBoss-EL analysis}

\section{Related Works}

Compared to other term rewriting based languages, like {\asfsdf}, {\maude},
{\elan}, {\stratego} an important advantage of {\tom} is its seamless
integration in any {\java} project. Other languages provide pattern-matching
extensions for {\java}: {\scala}, {\pizza}, {\jmatch}. To our knowledge, they
only provide a basic pattern-matching. More specifically, they lack the
list-matching, as well as the negative conditions. Other rule-based languages
like {\jrules}, {\jbossrules} or {\jess} have \emph{business rules} as their
application domain, and not program transformation.

The mapping concept proposed in~{\tom} is based on Philip
Wadler's\emph{views}~\cite{wadler87}. This concept gave rise to the {\pizza}
language~\cite{odersky97pizza} which is an extension of the {\java} language to
algebraic structures. In addition to the algebraic data types, this language
provided generics, closures and matching. The matching construction and the algebraic data
structures are less expressive than the language proposed by {\tom}
(for example, there is no matching modulo a theory and the data structures do not have
 maximal sharing or invariant). Generalist languages like
{\fsharp}~\cite{fsharp} and \scala~\cite{scala} offer primitives for the
definition of \emph{views}. These constructions are called respectively
\emph{active patterns} in {\fsharp}~\cite{syme07} and \emph{extractors} in
{\scala}~\cite{emir07}.  Defining views allow to deconstruct a same object in
different ways by changing the behaviour of the matching statement. This type
of views only allow to specify the destructive part of the mapping.  As for
the \emph{extractors} of {\scala}, {\tom} mappings allow in addition to
build data structures through the same abstractions used for matching.

Rule based languages (Elan, Stratego, Maude, \ldots)

Philip Wadler's views and Pizza.

F\# (active patterns), Scala (extractors), \ldots

\section{Conclusion}

\bibliographystyle{splncs}
\bibliography{paper}

\end{document}
