% !TEX root = ../shrink.tex

\noind Les deux \algos pr�c�dents ne r�duisent le contre-exemple qu'en consid�ration de son constructeur. C'est � dire qu'on ne profite de la r�duction que sur un seul type (celui du terme � r�duire). Pour r�duire �galement les autres types, on choisit de r�-appliquer les deux \algos sur les termes obtenus apr�s un premier passage, mais � des \textit{profondeurs}\footnote{La \textit{profondeur} d'un \st $t'$ d'un terme $t$ est la distance les s�parant.} diff�rentes. Cependant, le filtrage doit lui rester avec le m�me type (un filtrage sur un autre type n'aurait de toute fa�on pas de sens, car le filtrage est propre � un type).

\noind Concr�tement, on ajoute aux deux \algos \texttt{s1} et \texttt{s2} la notion de profondeur : si $t = c(t_1, t_2, \dots, t_n)$, alors on d�finit \texttt{s\textit{i}Depth} de la mani�re suivante :

$$ \forall d \in \N, \; \mathtt{s\mathit{i}Depth} (t,d) = 
	\left\{\begin{array}{cc}
		\mathtt{s}i(t) & \mbox{si $d = 0$} \\
		\displaystyle{\bigcup_{k \in \ent{1,n}}} \{c(t_1, \dots, t'_k, \dots, t_n)\}_{t'_k \in \mathtt{s\mathit{i}Depth(t_k,d-1)}}& \mbox{si $d \neq 0$}
	\end{array}\right.
$$


