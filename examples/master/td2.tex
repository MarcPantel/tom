\documentclass[11pt]{article}
\usepackage[utf8]{inputenc}
%% \usepackage{fullpage}
\usepackage{color}
\usepackage{aeguill}
\usepackage{comment}

%% \textwidth=5.5in
\textheight=8in

%% \includecomment{versionProf}
\specialcomment{versionProf}
    {\begingroup\bf}{\endgroup}

%% UNCOMMENT IF YOU WANT THE SHORT VERSION
\excludecomment{versionProf}

\title{Termes, réécriture, filtrage syntaxique}
\date{2007-2008}
\author{M1}


\begin{document}

%% \thispagestyle{empty}
%% \centerline{\LARGE\textbf{Termes, réécriture, filtrage syntaxique}}
%% \bigskip
%% \bigskip
\maketitle

Ce TD a pour but de vous familiariser avec le langage de programmation Tom.
Nous abordons dans ce TD l'utilisation de \verb|%gom|, \verb|%rule|, \verb|%match| et '\verb|`|'.

L'objectif est d'implanter un interpréteur ainsi qu'un optimiseur pour un pico-langage.

\section{Expression booléennes}
Dans un premier temps, on veut pouvoir construire et évaluer des expressions de la forme \verb|Not(True())|, \verb|And(Or(True(),False()),True())|, etc.

\begin{enumerate}
\item proposer une grammaire ainsi qu'un système de simplification.
\item simplifier \texttt{And(Or(Not(False),False),And(True,False))}
\end{enumerate}

\begin{versionProf}
\begin{verbatim}
    module Term
    imports 
    public
    sorts Inst Expr Bool
      
    abstract syntax
		True -> Bool
		False -> Bool
		Not(b:Bool) -> Bool
		Or(b1:Bool, b2:Bool) -> Bool
		And(b1:Bool, b2:Bool) -> Bool
\end{verbatim}

\begin{verbatim}
	public Bool evalBool(Map env,Bool bool) {
		%match(Bool bool) {	
			Not(True()) -> { return `False(); }
			Not(False()) -> { return `True(); }
			Not(b) -> { return `evalBool(env,Not(evalBool(env,b))); }

			Or(True(),b2) -> { return `True(); }
			Or(b1,True()) -> { return `True(); }
			Or(False(),b2) -> { return `b2; }
			Or(b1,False()) -> { return `b1; }
			And(True(),b2) -> { return `b2; }
			And(b1,True()) -> { return `b1; }
			And(False(),b2) -> { return `False(); }
			And(b1,False()) -> { return `False(); }
    + congruence pour Or et And

			x -> { return x; }
		}
	}

\end{verbatim}
\end{versionProf}

\section{Mini pico langage}
On considère un premier langage permettant de mémoriser des valeurs dans des variables. Sa grammaire est la suivante :
\begin{verbatim}
  <Inst> ::= Skip
           | Assign(<String>,<int>)
           | Print(<String>)
           | <Inst> ; <Inst>
\end{verbatim}

\begin{enumerate}
\item proposer une signature ainsi qu'une fonction permettant d'évaluer de tels programmes.
\end{enumerate}

\section{Pico langage}
On considère maintenant une extension de ce langage dont la grammaire est la suivante :

\begin{verbatim}
  <Inst> ::= Skip
           | Assign(<String>,<Exp>)
           | Print(<Exp>)
           | <Inst> ; <Inst>
           | If(<Bool>,<Inst>,<Inst>)
           | While(<Bool>,<Inst>)
  <Bool> ::= True
           | False
           | Not(<Bool>)
           | Or(<Bool>,<Bool>)
           | And(<Bool>,<Bool>)
           | Eq(<Expr>,<Expr>)
  <Expr> ::= Cst(<int>)
           | Var(<String>)
           | Plus(<Expr>,<Expr>)
           | Mult(<Expr>,<Expr>)
\end{verbatim}

\begin{enumerate}
\item écrire le programme \texttt{a:=1 ; b:=a+2 ; Print(b)} dans la syntaxe de Pico
\item écrire un programme qui affiche les entiers de 0 à 9.
\item donner la signature et les fonctions d'évaluation d'un tel langage
\end{enumerate}

\section{Optimisation}
On désire maintenant décrire une fonction qui élimine les instructions inutiles
(\texttt{Skip}) et optimise les tests de la forme \texttt{If(Not(...),...)} 

\section{Programmation}
\begin{enumerate}
\item écrire un programme qui calcule les nombres premiers jusqu'à 100
\item que faut-il ajouter comme instruction à Pico pour que cela soit plus facile
\end{enumerate}

\end{document}
