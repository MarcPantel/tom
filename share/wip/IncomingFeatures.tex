\documentclass{article}

\newcommand{\tom}{\textsc{Tom} }

\title{Incoming Features\\Work in Progress}
\author{Tom Team}
\date\today

\begin{document}
\maketitle

\tableofcontents

\section{Namespaces XML}
%%%%%%%%%%%%%%%%%%%%%%%%%%%%%%%%%%%%%%%%%%%%%%%%%%%%%%%%%%%%
It is desirable to deal with XML terms and patterns in \tom that contain namespace information, such as 
\verb$<n1:A></n1:A>$. Namespaces apply both to a XML document's elements and attributes. 

In the following example, the namespace prefix ``edi'' is bound to the URI ``http://ecommerce.org/schema'' (called namespace name). 

\begin{verbatim}
<x xmlns:edi='http://ecommerce.org/schema'>
  <!-- the 'price' element's namespace is http://ecommerce.org/schema -->
  <edi:price units='Euro'>32.18</edi:price>
</x>
\end{verbatim}

In this way we can declare a namespace inside a specific tag, whose scope comprises the element where it is specified and all elements within the content of that element, unless overridden by another namespace declaration. 

The use of a prefix before an element or attribute name specifies a qualified name.

\begin{verbatim}
QName 	 ::= 	(Prefix ':')? LocalPart
Prefix 	::= 	NCName
LocalPart 	::= 	NCName
\end{verbatim}


By their turn, qualified names appear in XML tags according to the grammar bellow:

\begin{verbatim}
 STag 	 ::= 	'<' QName (Attribute)* '>' 	 
 ETag 	::= 	'</' QName '>' 	
 EmptyElemTag 	::= 	'<' QName (Attribute)*  '/>' 	
 Attribute 	 ::= 	NSAttName Eq AttValue
			| QName Eq AttValue
\end{verbatim}

\subsection{Namespace Defaulting}

A default namespace is considered to apply to the element where it is declared (if that element has no namespace prefix), and to all elements with no prefix within the content of that element. If the URI reference in a default namespace declaration is empty, then unprefixed elements in the scope of the declaration are not considered to be in any namespace. An important remark is that default namespaces do not apply directly to attributes. See the examples bellow:

\begin{verbatim}
<?xml version="1.0"?>
<!-- elements are in the HTML namespace, in this case by default -->
<html xmlns='http://www.w3.org/TR/REC-html40'>
  <head><title>Frobnostication</title></head>
  <body><p>Moved to 
    <a href='http://frob.com'>here</a>.</p></body>
</html>
\end{verbatim}

\begin{verbatim}
<?xml version="1.0"?>
<!-- unprefixed element types are from "books" -->
<book xmlns='urn:loc.gov:books'
      xmlns:isbn='urn:ISBN:0-395-36341-6'>
    <title>Cheaper by the Dozen</title>
    <isbn:number>1568491379</isbn:number>
</book>
\end{verbatim}


\subsection{How it works for XQuery}

When element or attribute names are compared, they are considered identical if the local parts and namespace URIs match on a codepoint basis. Namespace prefixes need not be identical for two names to match, as illustrated by the following example:

\begin{verbatim}

declare namespace xx = "http://example.org";

let $i := <foo:bar xmlns:foo = "http://example.org">
              <foo:bing> Lentils </foo:bing>
          </foo:bar>
return $i/xx:bing
\end{verbatim}

Although the namespace prefixes xx and foo differ, both are bound to the namespace URI "http://example.org". Since xx:bing and foo:bing have the same local name and the same namespace URI, they match. The output of the above query is as follows.

\verb$<foo:bing xmlns:foo = "http://example.org"> Lentils </foo:bing>$

\subsection{How it should work in \tom}

If we take as basis the URI to which the prefix of a tag was bound, there is no much sense to use symbols to represent namespaces prefixes in XML patterns inside \tom programs. Mainly because the elements to be matched can have the same meaning, use a different prefix, but can be bound to the URI expected by the programmer. For example, the result of matching the  pattern \footnote{This example supposes a traversal scenario}  \verb$<title> x </title> -> ...$ against the following term:

\begin{verbatim}
<?xml version="1.0"?>
<!-- initially, the default namespace is "books" -->
<book xmlns='urn:loc.gov:books'
      xmlns:isbn='urn:ISBN:0-395-36341-6'>
    <title>Cheaper by the Dozen</title>
    <isbn:number>1568491379</isbn:number>
    <notes>
      <!-- make HTML the default namespace for some commentary -->
      <html xmlns='urn:w3-org-ns:HTML'>
        <head> <title> Book notes </title> </head>
        <body>
          <p >
             This is a <i>funny</i> book!
         </p>
       </body>
      </html> 
    </notes>
</book>
\end{verbatim}

In order to get a correct result, the user must be able to specify to which namespace the terms to be matched have to be bound in \tom. For the example above, the pattern should not match the term, since the element ``title'' is not bound to a empty namespace as default (it is possible to assign empty with the declaration \verb$xmlns=''$). 

In the case the user does not care about namespace associations, perhaps it is suitable to write something like \verb$<_:title> x </_:title> -> ...$. As result, both occurences of the node ``title'' would be filtered. On the other hand, It can be interesting for the user to capture sets of elements or attributes that belong to a certain namespace, consequently, the language has to provide a method to set the namespaces to be used in XML patterns. 

Addtionally, the \tom parser must be adapted to store the namespace information considering their respective scopes.

\section{Less restrictions with lists}
%%%%%%%%%%%%%%%%%%%%%%%%%%%%%%%%%%%%%%%%%%%%%%%%%%%%%%%%%%%%
\begin{verbatim}
symbol(domain*) -> codomain
\end{verbatim}
We want to define several constructors such as :
\begin{verbatim}
and(bool*) -> bool
or(bool*) -> bool
\end{verbatim}
Today, this limitation is due to Apigen which only allows one contructor per codomain for lists.
A solution is to implement a new tool $vas-to-java$ which stands in Apigen.

\section{Overloading}
%%%%%%%%%%%%%%%%%%%%%%%%%%%%%%%%%%%%%%%%%%%%%%%%%%%%%%%%%%%%
\subsection{Rules for overloading}
\begin{itemize}
\item Symbols with the same name should have the same codomain.
\item Likewise, slots with the same name should have the same domain.
\item When using the implicit notation (with []), the disjunction of overloaded functions only containts those who have all the slots used.
\item When typing with explicit notation, if a variable or an underscore could have more than one type (after resolution of constraints, op \dots), this may through an error.
\end{itemize}
\subsection{Types inference problem}
In order to avoid typing mistakes, brackets becames necessary for constants.
\subsection{Overloading in backquotes}
When there isn't any ambiguity, it still work the same way as before.
Otherwise, we have something like this :
\begin{verbatim}
f(s:A) -> B
f(s:B) -> B
\end{verbatim}
In \verb$`f(x)$, we can't predict wich $f$ should be use. It is necessary to cast the argument in \verb$`f((A)x)$ or \verb$`f((B)x)$. This cast will stay later in the host code.
\subsection{\%fun}
Before using Java function in algebric pattern, it's better to declare its profile with \verb$%fun$. It works as \%op and call the Java function in the make constructor.

\section{Backquote with implicit notation}
%%%%%%%%%%%%%%%%%%%%%%%%%%%%%%%%%%%%%%%%%%%%%%%%%%%%%%%%%%%%
\begin{verbatim}
f(s1:A,s2:A) -> B
g(s1:A,s2:A,s3:A) -> B
t@(f|g)[s2=a] -> `t@[s2=a]
              -> `t@f[s2=a]
\end{verbatim}
\subsection{Rules}
\begin{itemize}
\item \verb$`t@[s2=a]$ stands for \verb$`t@(f|g)[s2=a]$
\item Using a disjunction notation could be confusion. Indeed, \verb$t@(f|g|h)[s2=a] -> `t@[s2=a]$ cause a runtime error.
\item The $t$ must be intenciated by a $@$ before. Otherwise, there is a problem :
\begin{verbatim}
t@(f|g)[...] -> match(t) {
                     (f|h)[...] -> `t@f[...]
             }
\end{verbatim}
\item \verb$t@g[s2=a]$ is forbidden. Indeed, we could only build terms that implement each slots contained in the intersection of slots of terms of the disjunction (i.e. : f < g).
\end{itemize}
\subsection{With overloading}
It's in fact a hidden disjunction. All the slots have to be in the intersection of the generated disjunction For instance :
\begin{verbatim}
1. f(s1:A) -> B
2. f(s1:A,s2:A) -> B
3. f(s1:A,s2:A,s3:A) -> B

t@f[s2=a] -> `t@[s2=a,s3=a]
\end{verbatim}
The disjunction should be :
\begin{verbatim}
t@(f2|f3)[s2=a] -> `t@[s2=a,s3=a] ??? \\ not sure
\end{verbatim}
Explicit notation does not make sense.

\section{\%strategy}

We would like to add an operator \%strategy in Tom in order to write strategy easier.

\%strategy has this construction:
\begin{verbatim}
%strategy FindLeaves(bag:Collection) extends `Identity() {
  visit Term {
  F(_) -> {return ...}
  x    -> { if (x.getArity()) ... this ...}
  }
}
\end{verbatim}
it generates the following java class:
\begin{verbatim}

class FindLeaves extends strategy.term.termVisitableFwd {
Collection bag;
protected FindLeaves(Collection bag) {
  super(`Identity());
  this.bag = bag;
}
protected Term visit_Term (Term arg) throws VisitFailure {
  %match(Term arg) {
    F(_) -> {return ...}
    x    -> { if (x.getArity()) ... this ...}
  }
}
}
\end{verbatim}

Note that:
\begin{itemize}
\item{\%strategy can have arguments (here we pass "bag:Collection") to be able to pass variables to the corresponding java class.}
\item{The corresponding java class extends a parent class, here termVisitableFwd, and more generally visit\_TypeVisitableFwd, which implements interface VisitableVisitor. The super constructor takes 'Identity() as parameter. To hide this mechanism to Tom users, we just write constructor \%strategy extends 'Identity().}
\item{By default, we use java keyword 'private' for methods. To use 'protected', you must compile using option '--protected'.}
\end{itemize}

\section{constraints}
\dots

\end{document}
