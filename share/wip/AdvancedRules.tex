\documentclass{article}
%\usepackage{amsmath}

\newcommand{\tom}{\textsc{Tom} }
\newcommand{\filter}{\ll}

\title{Document de travail :\\ Impl\'ementation des contraintes dans \tom}
\author{Anne-Claire Lonchamp, Pierre-Etienne Moreau}
\date\today

\begin{document}
\maketitle

\section{Syntaxe des contraintes dans \tom}

\subsection{Actuellement}

Les conditions, ou gardes, ont \'et\'e introduites dans \tom r\'ecement.
\\

Actuellement les condition se pr\'esentent sous la forme de fonction (Java ou \tom) retournant un bool\'een.
Seuls des variables instenti\'e dans le pattern du \%match, ou du pattern filtrant pr\'ec\'edant dans le cas de \%match imbriqu\'es, peuvent \^etre pass\'es en arguments.
\\\\
$\%\:match\:\{$

$p\:when\:c\rightarrow\{\dots\}$

$p\:when\:c_1$\textasciicircum\dots\textasciicircum$ c_n\rightarrow\{\dots\}$
\\$\}$

\subsection{Proposition pour la nouvelle syntaxe}
$\%\:advanced\:\{$

$(p\filter s)$\textasciicircum$c\rightarrow\{\dots\}$

$(p_1\filter s_1)$\textasciicircum\dots\textasciicircum$(p_n\filter s_n)$\textasciicircum$c_1$\textasciicircum\dots\textasciicircum$c_n\rightarrow\{\dots\}$
\\$\}$
\\\\Deux cas de figures se pr\'esentent :
\begin{itemize}
\item $s_1\dots s_n$ sont ind\'ependants. On accroche les gardes simplement et la r\'esolution se fait dans l'ordre d'\'ecriture des patterns.
\item $s_1\dots s_n$ sont li\'es. L'ordre de r\'esolution doit cette fois ci \^etre d\'etermin\'e par un graphe de d\'ependances. Pour cette raison, les cycles de d\'ependances ne sont permis.
\end{itemize}

\subsection{Syntaxe proche de JRule}
$((\_^*,p,\_^*)\filter s)$\textasciicircum$c\rightarrow\{\dots\}$ devient : $(p\in E)$\textasciicircum$c\rightarrow\{\dots\}$
\\

Cette notation pourrait intervenir en tant qu'alternative syntaxique, sans engendrer de modification dans le traitement.

\section{Applications}

Cette nouvelle forme de r\'esolution des contraintes est particuli\`erement int\'eressante dans le cas o\`u l'on \'ecrit actuellement des \%match imbriqu\'es. En effet l'arbre des solutions \`a parcourir se r\'eduit.

\end{document}
