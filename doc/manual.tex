\documentclass[a4paper]{book}
\usepackage{fullpage}

\usepackage{alltt}
\usepackage{verbatim}
\usepackage{color}
\usepackage{url}
\usepackage{hevea}

\newcommand{\TOM}{\textsc{Tom}}
\newcommand{\JTOM}{\textsc{JTom}}
\newcommand{\Clang}{\textsf{C}}
\newcommand{\Cplusplus}{\textsf{C++}}
\newcommand{\Java}{\textsf{Java}}
\newcommand{\Eiffel}{\textsf{Eiffel}}

\newcommand{\typeterm}{\ahrefloc{typedef}{\lex{\textbf{\%typeterm}}}}
\newcommand{\op}{\ahrefloc{opdef}{\lex{\textbf{\%op}}}}
\newcommand{\typeint}{\ahrefloc{int}{\lex{\textbf{\%typeint}}}}
\newcommand{\typelist}{\ahrefloc{listdef}{\lex{\textbf{\%typelist}}}}
\newcommand{\oplist}{\ahrefloc{oplistdef}{\lex{\textbf{\%oplist}}}}
\newcommand{\typearray}{\ahrefloc{arraydef}{\lex{\textbf{\%typearray}}}}
\newcommand{\oparray}{\ahrefloc{oparraydef}{\lex{\textbf{\%oparray}}}}
\newcommand{\match}{\ahrefloc{match}{\lex{\textbf{\%match}}}}

\newcommand{\implement}{\ahrefloc{implement}{\lex{implement}}}
\newcommand{\getfs}{\ahrefloc{getfs}{\lex{get\_fun\_sym}}}
\newcommand{\cmpfs}{\ahrefloc{cmpfs}{\lex{cmp\_fun\_sym}}}
\newcommand{\getsub}{\ahrefloc{getsub}{\lex{get\_subterm}}}
\newcommand{\fsym}{\ahrefloc{fsym}{\lex{fsym}}}
\newcommand{\make}{\ahrefloc{make}{\lex{make}}}
\newcommand{\backquote}{\ahrefloc{backquote}{\lex{\textbf{`}}}}
\newcommand{\isfs}{\ahrefloc{isfsym}{\lex{is\_fsym}}}
\newcommand{\getslot}{\ahrefloc{getslot}{\lex{get\_slot}}}
\newcommand{\equals}{\ahrefloc{equals}{\lex{equals}}}
\newcommand{\geth}{\ahrefloc{geth}{\lex{get\_head}}}
\newcommand{\gett}{\ahrefloc{gett}{\lex{get\_tail}}}
\newcommand{\isempty}{\ahrefloc{isempty}{\lex{is\_empty}}}
\newcommand{\getel}{\ahrefloc{gete}{\lex{get\_element}}}
\newcommand{\getsize}{\ahrefloc{getsize}{\lex{get\_size}}}
\newcommand{\emptyl}{\ahrefloc{emptylist}{\lex{make\_empty}}}
\newcommand{\emptya}{\ahrefloc{emptyarray}{\lex{make\_empty}}}
\newcommand{\minsert}{\ahrefloc{insert}{\lex{make\_insert}}}
\newcommand{\mappend}{\ahrefloc{append}{\lex{make\_append}}}

\newcommand{\alt}{$\mid$}
\newcommand{\lex}[1]{\textcolor{blue}{\texttt{#1}}}
\newcommand{\nt}[1]{\textcolor{magenta}{\textit{#1}}}

\urldef{\tompage}{\url}{http://tom.loria.fr}
\urldef{\packagebase}{\url}{http://www.cwi.nl/htbin/sen1/twiki/bin/view/SEN1/PackageBase}
\urldef{\atermpage}{\url}{http://www.cwi.nl/htbin/sen1/twiki/bin/view/SEN1/ATermLibrary}
\urldef{\apigenpage}{\url}{http://www.cwi.nl/htbin/sen1/twiki/bin/view/SEN1/ApiGen}
\urldef{\manualpage}{\url}{http://tom.loria.fr/doc/manual/manual.html}
\urldef{\manualPS}{\url}{http://tom.loria.fr/doc/manual/manual.ps}
\urldef{\manualPDF}{\url}{http://tom.loria.fr/doc/manual/manual.pdf}

\title{\TOM\ Language Reference}
\author{Pierre-Etienne Moreau\\(with Julien Guyon and Christophe Ringeissen)}
\date{September 29, 2003}
\setcounter{tocdepth}{1}
\begin{document}
\maketitle

\begin{table}[h]
This manual also exists in  \ahref{\manualPS}{Postscript} or \ahref{\manualPDF}{pdf}.
\end{table}

\textbf{Abstract:}~TOM is a Pattern Matching Preprocessor that can be used to
intregrate term rewriting facilities in an imperative language such as
\Clang\ and \Java.

Information on TOM is available the TOM web page~\ahref{\tompage}{tom.loria.fr}.

\tableofcontents

%\newpage
%%%%%%%%%%%%%%%%%%%%%%%%%%%%%%%%%%%%%%%%%%%%%%%%%%%%%%%%%%%%%
\section{Introduction}
%%%%%%%%%%%%%%%%%%%%%%%%%%%%%%%%%%%%%%%%%%%%%%%%%%%%%%%%%%%%

This manual describes the functionality provided by \TOM. 
This tool is a Pattern Matching Preprocessor that can be used to
intregrate term rewriting facilities in an imperative language such as
\C\ and \Java.

%%%%%%%%%%%%%%%%%%%%%%%%%%%%%%%%%%%%%%%%%%%%%%%%%%%%%%%%%%%%
\section{Using \TOM}
%%%%%%%%%%%%%%%%%%%%%%%%%%%%%%%%%%%%%%%%%%%%%%%%%%%%%%%%%%%%

This section explains the basics of using \TOM. 

\subsection{Installing \TOM}
%%%%%%%%%%%%%%%%%%%%%%%%%%%%%%%%%%%%%%%%%%%%%%%%%%%%%%%%%%%%
In order to run \TOM, you need a Java Development Kit (JDK)
%\TOM\ is written in \JTOM\ (\TOM\ $+$ \Java), 

\subsection{Programming in \TOM}
%%%%%%%%%%%%%%%%%%%%%%%%%%%%%%%%%%%%%%%%%%%%%%%%%%%%%%%%%%%%
\TOM\ is a multi-languages Pattern Matching Preprocesor. It currently
supports a least two languages (called ``target languages''): \C\ and \Java.
A \TOM\ program can be seen a \Java\ program (resp. a \C\ program)
that contains some \TOM\ constructs. As consequence, any \Java\ or \C\
program is a valid \TOM\ program. 

\TOM\ provides several constructs for specifying algebraic data types
and associated pattern matching operations, and produces as output the
same code with all such constructs translated into the target language.
As an example, we consider the algebraic specification of Naturals,
using the Peano axiomatisation:
% and will we show how to make this
% specification executable using \TOM: 

$$
Nat = 0 \mid suc(Nat) \mid plus(Nat,Nat)
$$
with the two following rewrite rules:
$$
\begin{array}{lcl}
  plus(x,0)      & \rightarrow & x\\
  plus(x,suc(y)) & \rightarrow & suc(plus(x,y))
\end{array}
$$
(where $x$ and $y$ are variables of sort~$Nat$)

\medskip
Suppose that we want to write a \C\ program that makes this
specification executable. 
Using \TOM, the program could look like:

\begin{verbatimwrite}{program.txt}
  %typeterm Nat
  %op Nat zero
  %op Nat suc(Nat)
  %op Nat plus(Nat,Nat)

  %rule {
     plus(x,0)      -> x
     plus(x,suc(y)) -> suc(plus(x,y))
  }

  int main() {
    ...
    result = plus( suc(suc(zero)) , suc(zero) )
    ...
  }
\end{verbatimwrite}
\programboxed{program.txt}


In fact, this previous part of program is not complete. As we can
notice, we need to provide a term data structure to represent the
objects we want to rewrite.
To achieve this goal, we can either use and external term library such
as the ATerm library, either we have to explicitly describe how a term
is represented in memory.
In \C\, this can be done as follow:
\begin{verbatimwrite}{program.txt}
  struct term {
    int symbol;
    int arity;
    struct term **subterm;
  };

  #define ZERO 0
  #define SUC 1
  #define PLUS 2
\end{verbatimwrite}
\programboxed{program.txt}

Then, we can provide some function to build a term:

\begin{verbatimwrite}{program.txt}
  struct term symbol_zero = {ZERO, 0, NULL};
  struct term *zero = &symbol_zero;

  struct term *suc(struct term *x) {
    struct term *res;
    res = malloc(sizeof(struct term));
    res->symbol = SUC;
    res->arity = 1;
    res->subterm = (struct term **) malloc(1 * sizeof(struct term *));
    res->subterm[0] = x;
    return(res);
  }
\end{verbatimwrite}
\programboxed{program.txt}

Using this term representation, the \C\ expressions 
\texttt{suc(suc(suc(zero)))} and \texttt{suc(suc(zero))} build the
naturals~3 and~2.
But, we still have to ``explain'' to \TOM\ how to access to our term
representation: for each data type defined in \TOM, we have to specify
how this data type is implemented, how to access to the function
symbol, how to access to a subterm, and how to compare two function symbols.
Then, for each constructor of this data type, we also have to specify
how the constructor is represented in memory, and how to build such a constructor.
This can be done in the following way:

\begin{verbatimwrite}{program.txt}
  %typeterm Nat {
    implement { struct term* }
    get_fun_sym(t)      { t->symbol }
    cmp_fun_sym(s1,s2)  { s1 == s2 }
    get_subterm(t, n)   { t->subterm[n] }
  }
  %op term zero {
    fsym { ZERO }
    make { zero } 
  }
  %op term suc(term) {
    fsym    { SUC }
    make(t) { suc(t) }
  }
  %op term plus(term,term) {
    fsym { PLUS }
  }
\end{verbatimwrite}
\programboxed{program.txt}

Given the previous specification, \TOM\ will generate a \C\ function
\texttt{struct term *plus(struct term *t1, struct term *t2)} that
implements the corresponding rewrite system. In this example, $plus$ is
a defined symbol: this explains why we do not have to specify how to
build such a symbol.

Instead of using the \texttt{\%rule} construct, another alternative
could have consisted in using the \texttt{\%match} construct. 
This more primite construct may remind the \texttt{switch-case}
\C~instruction. It can be used as follow:

\begin{verbatimwrite}{program.txt}
  struct term *plus(struct term *t1, struct term *t2) {
    %match(term t1, term t2) {
      x,zero   -> { return x; }
      x,suc(y) -> { return suc(plus(x,y)); }
    }
  }
\end{verbatimwrite}
\programboxed{program.txt}

The \texttt{plus} function is now explicitly defined by the user, and
the \texttt{\%match} construct is now replaced by several \C\
instructions (and not a function definition). Another difference,
wrt. the \texttt{\%rule} construct, is related to the right-hand side
of each rule: 
\begin{itemize}
\item when using \texttt{\%rule}, the right-hand side is a
  term built on the algebraic data type
\item when using \texttt{\%match}, the right-hand side is a list of
  target language instructions. In the second rule of our example, we
  explicitly call the \C\ functions \texttt{plus} and \texttt{suc} and
  we return the resulting term, but we can imagine a program that
  performs input/output or side effects that cannot be easilly
  expressed in a pure algebraic specification formalism. 
\end{itemize}


%\newpage
%%%%%%%%%%%%%%%%%%%%%%%%%%%%%%%%%%%%%%%%%%%%%%%%%%%%%%%%%%%%
\part{Installing and using the system}
%%%%%%%%%%%%%%%%%%%%%%%%%%%%%%%%%%%%%%%%%%%%%%%%%%%%%%%%%%%%
\chapter{Installing and using the system}
\cutname{install.html}
%%%%%%%%%%%%%%%%%%%%%%%%%%%%%%%%%%%%%%%%%%%%%%%%%%%%%%%%%%%%
This part of the manual gives basic information to get and install the
system.

\section{Requirements and download}
%%%%%%%%%%%%%%%%%%%%%%%%%%%%%%%%%%%%%%%%%%%%%%%%%%%%%%%%%%%%
The current version of \TOM\ is available at \ahref{\tompage}{TOM Home Page}.
In this page, you will find the latest release of \TOM.
\\There are 2 kinds of packages:
\begin{itemize}
\item \TOM\ software package;
\item A bundle package with \TOM\ software itself and all dependencies
  packages;
\end{itemize}

\paragraph{Compiling and runtime tools}
\TOM\ is written in Java and \TOM\ itself. You will need a Java
compiler and interpreter to compile and run \TOM. 
\\We actually use Sun JDK 1.3.1 and 1.4 tools for development and runtime
excecution.
\\Some tests have already been realized with gcj
and Jikes without any particular issues.

\paragraph{Package dependencies}
\TOM\ software depends on others packages:
\begin{itemize}
\item shared-objects
\item JJTraveler
\item aterm-java
\end{itemize}
The following image shows the dependencies order:

\imgsrc{depgraphjtom.pgn}

Such packages are available at:
\begin{itemize}
\item \ahref{\tompage}{\TOM\ Home Page}
\item \ahref{\packagebase}{CWI package base Page}
\end{itemize}    

\section{Installation}
%%%%%%%%%%%%%%%%%%%%%%%%%%%%%%%%%%%%%%%%%%%%%%%%%%%%%%%%%%%%
Depending on the\TOM\ package you download, here is the installation
instructions.

\subsection{Installing the bundle package}
This is the simplest way to have quickly \TOM\ running on your
computer.
\begin{itemize}
\item Untar the package: 
  > tar xzf jtom-bundle-VERSION.tar.gz
\item Download and unpack the required packages:
  \\> ./collect.sh
\item Configure, build and install:
  \\> ./configure --help gives a complete list of available configuration parameters.
  \\> ./configure <<configuration parameters>>
  \\> gmake
\end{itemize}

Once all is done, you are ready using \TOM\ using jtom command.

\subsection{Installing \TOM\ package}

\paragraph{Installation process}
\begin{enumerate}
\item Get and install the shared-objects package
\\ a. Type `./configure --prefix=<SHARED-DIR>'
\\ b. Type `make'
\\ c. Type `make install'
\item Get and install the JJTraveler package
\\ a. Type `./configure --prefix=<JJT-DIR>'
\\ b. Type `make'
\\ c. Type `make install'
\item Get and install the aterm-java package
\\ a. Type `./configure --prefix=<ATERM-DIR>
--with-JJTraveler=<JJT-DIR> --with-shared-objects=<SHARED-DIR>' 
\\ b. Type `make'
\\ c. Type `make install'

\item Configure and install \TOM\ package:
\\ a./configure --prefix=<INSTALLDIR>
--with-shared-objects=<SHARED-DIR> --with-aterm-java=<ATERM-DIR>'
\\ b. Type `make' to compile the package.
\\ c. Type `make install' to install the package.
\end{enumerate}


\section{Command line Arguments}
%%%%%%%%%%%%%%%%%%%%%%%%%%%%%%%%%%%%%%%%%%%%%%%%%%%%%%%%%%%%
\subsection{NAME}
     jtom - compile \TOM\ file (.t) into specified target languages

\subsection{SYNOPSIS}
     jtom [-hceVvioDCfWldpsO] [-I path] filename[.t]

\subsection{DESCRIPTION}
\begin{tabular}{|l|l|}
\hline
--help | -h&            Show the help \\
--cCode | -c&           Generate C code (default is Java)\\
--eCode | -e&           Generate Eiffel code (default is Java)\\
--version | -V&         Print the version of TOM\\
--verbose | -v&         Set verbose mode on: give duration information
on each compilation passes\\
--intermediate | -i&    Generate intermediate files \\
--noOutput | -o&        Do not generate code \\
--noDeclaration | -D&   Do not generate code for declarations \\
--doCompile | -C&       Start after type-checking (used after a
compilation process with --intermediate option\\
--noCheck | -f&         Do not realize checking phases \\
--noWarning | -W&       Do not print any warning \\
--lazyType | -l&        Use universal type \\
--demo | -d&            Run demo mode \\
--import <path> | -I&   Path for \%include construct to find included files\\
--pretty | -p&          Generate readable code with indentation \\
--atermStat | -s&       Print internal ATerm statistics \\
--optimize | -O&        Optimized generated code \\
--static&               Generate static functions \\
--debug&                Generate debug primitives \\
\hline
\end{tabular}


\section{For developpers}\label{developpers}
%%%%%%%%%%%%%%%%%%%%%%%%%%%%%%%%%%%%%%%%%%%%%%%%%%%%%%%%%%%%
\paragraph{CVS repository}
The latest developments of \TOM\ are available from anonymous cvs at:
\\cvs -d :pserver:cvs@cvs-sop.inria.fr:/CVS/aircube checkout jtom
\paragraph{\TOM\ Architecture}
In \TOM\ directory architecture, 2 main branches can be underlined:
`stable` and `src`...
The `stable` branch contains the stable material to build a running
\TOM\ compiler. The `src` branch is used to make new developments.
Once in src directory, you can modify the sources. then, 
\begin{enumerate}
\item Type `make' to build a new system and use jtom.src to use and
  test the new system.
\item At `src` directory level, type `make bootstrap' to bootstrap and
  install a new system.
  If something goes wrong, re-install the stable version:
  `cd ../../stable'; `make install'
  Then, correct the `src' directory
\item If you are happy from the result, type `make bootinstall' to
  make the current version become the stable one.   WARNING: this action may be VERY DANGEROUS !!!
\end{enumerate}
Please send a mail to \mailto{pierre-etienne.moreau@loria.fr} to propose your
suggestions.

\paragraph{Additional tools for developments}
To simplify \TOM\ development, an extra package called Apigen is used
to generate data structures used by \TOM. All generated files can be
found in `jtom $\backslash$adt` directory.
\\This package is available at
\ahref{\packagebase}{CWI package base Page}.
WARNING: We will need to install JUnit to build this package.
Once TomSignature.adt is modified, `make api` to re-generate \TOM\
data structures.
 


%\newpage
%%%%%%%%%%%%%%%%%%%%%%%%%%%%%%%%%%%%%%%%%%%%%%%%%%%%%%%%%%%%
\section{The system}
%%%%%%%%%%%%%%%%%%%%%%%%%%%%%%%%%%%%%%%%%%%%%%%%%%%%%%%%%%%%

The current version of \TOM\footnote{available at
  \texttt{http://www.loria.fr/ELAN/Toolkit}} is written in
\Java+\TOM\ itself. 
It reads the program to be compiled and builds an abstract syntax tree
(AST) to represent the program.
The compiler is made up of stages, each of which can be seen as a
process that transforms the AST into a new one. After the last transformation,
the AST reprents a program closed to an imperative program.
The last stage of the compiler is a generation phase that transform
the AST into a concrete program written in \Clang\ or \Java, depending 
on the chosen target language.

%\subsection{How \TOM\ is implemented}
%%%%%%%%%%%%%%%%%%%%%%%%%%%%%%%%%%%%%%%%%%%%%%%%%%%%%%%%%%%%

\subsection{The front-end}
%%%%%%%%%%%%%%%%%%%%%%%%%%%%%%%%%%%%%%%%%%%%%%%%%%%%%%%%%%%%
The front-end is the part of the system that reads the input program
and builds the associated abstract syntax tree.
This parts contains three components:
\begin{itemize}
\item a parser: this stage reads a concrete program written in the
  target language embedded with some \TOM\ constructs, and build a
  first AST.
\item an expander: this stage performs some very simple
  transformations such as macro expansion.
\item a type checker: this stage add some type information to the AST
  (all untyped vaiables become a typed variable for example).
\end{itemize}

Because \TOM\ is language independant, parsing a \TOM\ program is not
so easy. A solution could consists in implementing a specialized
parser for each supported target language (one for \Clang+\TOM, and
another one for \Java+\TOM), but for simplicity, and to keep the \TOM\
system as simple as possible, we decided to implement a common parser,
slightly specialised for each considered target language. 
When considering \Clang\ and \Java, we noticed that it is only needed to
know how to recognize a string, a comment and a block to be able to
make the difference between a target language construct and a \TOM
construct.

The main idea consists in synchronising the parser on several
characters such as `\texttt{\%}',`\texttt{"}', `\texttt{\{}' and
`\texttt{\}}'. For this purpose we decided to use \textsf{Javacc} and
the lexical-mode facilities. Basically, the parser can be in two
differents modes: \texttt{TomConstruct} mode and
\texttt{TargetLanguage} mode.
When being in the \texttt{TargetLanguage} mode, the parser reads
everything unless a \TOM\ construct (beginning with a `\texttt{\%}'
character) is recognized. Of course, this construct should not be in a
target language string or comment. This explain why it is needed to be
able to recognized such target language constructs.
Once a \TOM\ construct is recognized, the parser is switched to the
\texttt{TomConstruct} mode, and the considered construct can be
easilly parsed. 
We should notice that a \TOM\ construct can also contains a target
language part (always between `\texttt{\{}' and `\texttt{\}}').
When parsing such a part, the parser first reads a `\texttt{\{}', and
then is looking for a corresponding `\texttt{\}}': for each
encountered~`\texttt{\}}', it has to know if the read expression is
well parenthesed or not. This explains why it is needed to be able to
recognize and count the target language open and close block
commands. 

\subsection{The compiler}
%%%%%%%%%%%%%%%%%%%%%%%%%%%%%%%%%%%%%%%%%%%%%%%%%%%%%%%%%%%%
The compiler receives as input an AST that merely corresponds to the 
input \TOM\ program (this is roughly a list of interleaved target
language constructs and \TOM\ constructs).
The goal of the compiler consists in transforming this AST into a
``simpler one'': an AST that can be easilly translated into an
imperative program (a \Clang\ or a \Java\ program for example).

The kernel of the compiler contains a procedure that transforms a set
of patterns into a automaton that implements a matching algorithm
corresponding to the considered set of patterns. 
This automaton is then compiled into abstract instructions of the
form: \texttt{IfThenElse}, \texttt{Assign}, \texttt{ExitAction}, 
\texttt{ExecuteAction}, etc. 
With such an approach, the most complex part of the compilation process
is completly independ from the chosen target language, and can easily
be reused.


\subsection{The back-end}
%%%%%%%%%%%%%%%%%%%%%%%%%%%%%%%%%%%%%%%%%%%%%%%%%%%%%%%%%%%%
The back-end takes the last form of AST as input and generates a
concrete program written in the target language. This stage consists
in translating abstract instructions (like \texttt{IfThenElse}) into
concrete instructions (like \texttt{if(cond) \{ instList \}}) in \Clang\
or \Java.  
With such an approach, allows us to easily add a new back-end for any 
new supported target language.




%\newpage

%%%%%%%%%%%%%%%%%%%%%%%%%%%%%%%%%%%%%%%%%%%%%%%%%%%%%%%%%%%%
\section{The language}
%%%%%%%%%%%%%%%%%%%%%%%%%%%%%%%%%%%%%%%%%%%%%%%%%%%%%%%%%%%%

\subsection{Foreword}
%%%%%%%%%%%%%%%%%%%%%%%%%%%%%%%%%%%%%%%%%%%%%%%%%%%%%%%%%%%%
This document is intended as a reference manual for the \TOM\
language. It lists the language constructs, and gives their precise
syntax and informal semantics. It is by no means a tutorial
introduction to the language: there is not a single example.

\subsection{Notations}
%%%%%%%%%%%%%%%%%%%%%%%%%%%%%%%%%%%%%%%%%%%%%%%%%%%%%%%%%%%%

The syntax of the language is given in BNF-like notation. Terminal
symbols are set in typewriter font (\lex{like this}). Non-terminal
symbols are set in italic font (\nt{like  that}). Square brackets
[...] denote optional components. Parentheses with a trailing
star sign  (...)* denotes zero, one or several repetitions of the
enclosed components. Parentheses with a trailing 
plus sign  (...)+ denote one or several repetitions of the enclosed
components. Parentheses (...) denote grouping. 

\subsection{Lexical conventions}
%%%%%%%%%%%%%%%%%%%%%%%%%%%%%%%%%%%%%%%%%%%%%%%%%%%%%%%%%%%%
\begin{center}
\begin{tabular}{lcl}
  \nt{Identifier} & ::= & \nt{Letter} ( \nt{Letter} \alt \nt{Digit}
  \alt \lex{.} \alt \lex{\_} \alt \lex{-} )*\\
  \nt{Integer} & ::= & \nt{Digit} ( \nt{Digit} )*\\
  \nt{Letter} & ::= & \lex{A} ... \lex{Z} \alt \lex{a} ... \lex{z}\\
  \nt{Digit} & ::= & \lex{0} ... \lex{9}\\
  \nt{Other} & ::= & a character
\end{tabular}
\end{center}


\subsection{Names}
%%%%%%%%%%%%%%%%%%%%%%%%%%%%%%%%%%%%%%%%%%%%%%%%%%%%%%%%%%%%
\begin{center}
\begin{tabular}{lcl}
  \nt{SubjectName} & ::= & \nt{Identifier}\\
  \nt{Type} & ::= & \nt{Identifier}\\
  \nt{SlotName} & ::= & \nt{Identifier}\\
  \nt{SymbolName} & ::= & \nt{Identifier}\\
  \nt{VariableName} & ::= & \nt{Identifier}\\
  \nt{AnnotedName} & ::= & \nt{Identifier}\\
  \nt{FileName} & ::= & \nt{Identifier}\\
  \nt{Name} & ::= & \nt{Identifier}
\end{tabular}
\end{center}


\subsection{\TOM\ syntax}
%%%%%%%%%%%%%%%%%%%%%%%%%%%%%%%%%%%%%%%%%%%%%%%%%%%%%%%%%%%%
A \TOM\ program is a target language program (namely \Clang\ or \Java)
embedded with some new preprocessor constructs such as
\lex{\%typeterm}, \lex{\%op}, \lex{\%rule} and \lex{\%match}.
\TOM\ is a multi-languages preprocessor, so, its syntax depends from 
the target language syntax. But for simplicity, we will only present
the syntax of its constructs and explain how they can be integrated
into the target language.
Basically, a \TOM\ program is list of blocks, where each block is
either a \TOM\ construct, either a sequence of characters.
The idea is that that after transformation, the sequence of characters
merged with the compiled \TOM\ constructs should be a valid target
language program.
So we have:

\begin{center}
\begin{tabular}{lcl}
  \nt{Tom} & ::= & \nt{BlockList}\\
  \nt{BlockList} & ::= \\
  & ( &  \nt{MatchConstruct}\\
  & \alt & \nt{RuleConstruct}\\
  & \alt & \nt{BackQuoteTerm}\\
  & \alt & \nt{IncludeConstruct}\\
  & \alt & \nt{LocalVariableConstruct}\\
  & \alt & \nt{Operator}\\
  & \alt & \nt{OperatorList}\\
  & \alt & \nt{OperatorArray}\\
  & \alt & \nt{TypeTerm}\\
  & \alt & \nt{TypeInt}\\
  & \alt & \nt{TypeList}\\
  & \alt & \nt{TypeArray}\\
  & \alt & \lex{\{} \nt{BlockList} \lex{\}}\\
  & \alt & \nt{Other}\\
  &)*\\
\end{tabular}
\end{center}

\begin{itemize}
\item a \nt{MatchConstruct} is translated into a list of
  instructions. This constructs may appear anywhere a list of
  instructions is valid in the target language.

\item a \nt{RuleConstruct} is translated into a function
  definition. This constructs may appear anywhere function declaration
  is valid in the target language. 

\item a \nt{BackQuoteTerm} is translated into a function call.

\item a \nt{IncludeConstruct} is replaced by the content of the
  file referenced by the construct.

\item a \nt{LocalVariableConstruct} is replaced by variable declarations.

\item \nt{Operator}, \nt{OperatorList} and \nt{OperatorArray} are
  replaced by some functions definitions. 
 
\item \nt{TypeTerm}, \nt{TypeInt}, \nt{TypeList} and \nt{TypeArray}
  are also replaced by some functions definitions. 
\end{itemize}

A \lex{\%match} construct contains two parts:
\begin{itemize}
\item a list of target language variables. These variables should
  contains the object on which patterns are matched

\item a list of rules: a pattern and a semantic action (written in the
  target language)
\end{itemize}
The construct is defined as follow:

\begin{center}
  \begin{tabular}{lcl}
    \nt{MatchConstruct} & ::= & \lex{\%match} \lex{(}
    \nt{MatchArguments} \lex{)} \lex{\{}
    ( \nt{PatternAction} )* 
    \lex{\}}\\
  \nt{MatchArguments} & ::= & \nt{MatchArgument} ( \lex{,} \nt{MatchArgument} )* \\
  \nt{MatchArgument}  & ::= & \nt{Type} \nt{SubjectName}\\
  \nt{PatternAction} & ::= & \nt{MatchPatterns} \lex{->}
     \lex{\{} \nt{BlockList} \lex{\}} \\
  \nt{MatchPatterns} & ::= & \nt{Term} ( \lex{,} \nt{Term} )*
\end{tabular}
\end{center}

A term has the following syntax:
\begin{center}
  \begin{tabular}{lcl}
    \nt{Term} & ::= & [ \nt{AnnotedName} \lex{@} ] \nt{PlainTerm}\\
    \nt{PlainTerm} & ::= & \nt{SymbolName} \lex{[} \nt{SlotName} \lex{=} \nt{Term}
    ( \lex{,} \nt{SlotName} \lex{=} \nt{Term} )* \lex{]}\\
    & \alt & \nt{VariableName} \lex{*}\\
    & \alt & \nt{SymbolName} \lex{(} \nt{Term} ( \lex{,} \nt{Term} )* \lex{)}\\
    & \alt & \lex{\_}
  \end{tabular}
\end{center}

In \TOM, we can also define a set of rewrite rules. All the
left-handsides should begin with the same root symbol:

\begin{center}
  \begin{tabular}{lcl}
    \nt{RuleConstruct} & ::= & \lex{\%rule} \lex{\{} ( \nt{Term} \lex{->} \nt{Term} )* \lex{\}}\\
    \nt{MakeConstruct} & ::= & \lex{\%make} \lex{\{} \nt{Term} \lex{\}}
\end{tabular}
\end{center}

\begin{center}
  \begin{tabular}{lcl}
    \nt{IncludeConstruct} & ::= & \lex{\%include} \lex{\{} \nt{FileName} \lex {\}}\\
    \nt{Operator} & ::= & \lex{\%op} \nt{Type} \nt{Name}
        [ \lex{(} [ \nt{SlotName} \lex{:} ] \nt{Type} ( \lex{,} [
        \nt{SlotName} \lex{:} ] \nt{Type} )* \lex{)} ]\\
        &&
        \lex{\{} \nt{KeywordFsym} ( \nt{KeywordMake} \alt
        \nt{KeywordGetSlot} )* \lex{\}}\\
        \nt{OperatorList} & ::= & 
        \lex{\%oplist} \nt{Type} \nt{Name} \lex{(} \nt{Type} \lex{*}
        \lex{)}\\
        &&
        \lex{\{} \nt{KeywordFsym} ( \nt{KeywordMakeEmptyList} \alt
        \nt{KeywordMakeAddList} )* \lex{\}}\\
        \nt{OperatorArray} & ::= & 
        \lex{\%oparray} \nt{Type} \nt{Name} \lex{(} \nt{Type} \lex{*}
        \lex{)}\\
        &&
        \lex{\{} \nt{KeywordFsym} ( \nt{KeywordMakeEmptyEmpty} \alt
        \nt{KeywordMakeAddArray} )* \lex{\}}
\end{tabular}
\end{center}

\begin{center}
  \begin{tabular}{lcl}
    \nt{TypeTerm} & ::= & \lex{\%type} \nt{Type} \lex{\{}
    \nt{KeywordImplement}
    ( 
    \nt{KeywordGetFunSym} 
    \alt \nt{KeywordGetSubterm} 
    \alt \nt{KeywordCmpFunSym} 
    \alt \nt{KeywordEquals} 
    )*
    \lex{\}}\\
    \nt{TypeList} & ::= &
        \lex{\%typelist} \nt{Type} \lex{\{}
                \nt{KeywordImplement}
                [\nt{KeywordGetFunSym}]
                [\nt{KeywordGetSubterm}]
                [\nt{KeywordCmpFunSym}]
                [\nt{KeywordEquals}]
                [\nt{KeywordGetHead}]
                [\nt{KeywordGetTail}]
                [\nt{KeywordIsEmpty}]
                \lex{\}}\\
    \nt{TypeArray} & ::= &
        \lex{\%typearray} \nt{Type} \lex{\{}
                \nt{KeywordImplement}
                ( 
                  \nt{KeywordGetFunSym}
                \alt \nt{KeywordGetSubterm}
                \alt \nt{KeywordCmpFunSym}
                \alt \nt{KeywordEquals}
                \alt \nt{KeywordGetElement}
                \alt \nt{KeywordGetSize}
                )*
        \lex{\}}
\end{tabular}
\end{center}

\begin{center}
  \begin{tabular}{lcl}
\nt{GoalLanguageBlock} & ::= & \lex{\{} \nt{BlockList} \lex{\}}\\
\nt{KeywordImplement} & ::= & \lex{implement} \nt{GoalLanguageBlock}\\
\nt{KeywordGetFunSym} & ::= & \lex{get\_fun\_sym} \lex{(} \nt{Name} \lex{)} \nt{GoalLanguageBlock}\\
\nt{KeywordGetSubterm} & ::= & 
\lex{get\_subterm} \lex{(} \nt{Name} \lex{,} \nt{Name} \lex{)} \nt{GoalLanguageBlock}\\
\nt{KeywordCmpFunSym} & ::= & 
\lex{cmp\_fun\_sym} \lex{(} \nt{Name} \lex{,} \nt{Name} \lex{)} \nt{GoalLanguageBlock}\\
\nt{KeywordEquals} & ::= &
        \lex{equals} \lex{(} \nt{Name} \lex{,} \nt{Name} \lex{)} \nt{GoalLanguageBlock}\\
\nt{KeywordGetHead} & ::= &
        \lex{get\_head} \lex{(} \nt{Name} \lex{)} \nt{GoalLanguageBlock}\\
\nt{KeywordGetTail} & ::= &
        \lex{get\_tail} \lex{(} \nt{Name} \lex{)} \nt{GoalLanguageBlock}\\
\nt{KeywordIsEmpty} & ::= &
        \lex{is\_empty} \lex{(} \nt{Name} \lex{)} \nt{GoalLanguageBlock}\\
\nt{KeywordGetElement} & ::= &
        \lex{get\_element} \lex{(} \nt{Name} \lex{,} \nt{Name} \lex{)} \nt{GoalLanguageBlock}\\
\nt{KeywordGetSize} & ::= &
        \lex{get\_size} \lex{(} \nt{Name} \lex{)} \nt{GoalLanguageBlock}\\
\nt{KeywordFsym} & ::= &
        \lex{fsym} \nt{GoalLanguageBlock}\\
\nt{KeywordGetSlot} & ::= &
        \lex{get\_slot} \lex{(} \nt{Name} \lex{,} \nt{Name} \lex{)} \nt{GoalLanguageBlock}\\
\nt{KeywordMake} & ::= &
        \lex{make} \lex{(} \nt{Name} ( \lex{,} \nt{Name} )* \lex{)} \nt{GoalLanguageBlock}\\
\nt{KeywordMakeEmptyList} & ::= &
        \lex{make\_empty} [ \lex{(} \lex{)} ] \nt{GoalLanguageBlock}\\
\nt{KeywordMakeAddList} & ::= &
        \lex{make\_add} \lex{(} \nt{Name} \lex{,} \nt{Name} \lex{)} \nt{GoalLanguageBlock}\\
\nt{KeywordMakeEmptyArray} & ::= &
        \lex{make\_empty} \lex{(} \nt{Name} \lex{)} \nt{GoalLanguageBlock}\\
\nt{KeywordMakeAddArray} & ::= &
        \lex{make\_add} \lex{(} \nt{Name} \lex{,} \nt{Name} \lex{,} \nt{Name} \lex{)} \nt{GoalLanguageBlock}
\end{tabular}
\end{center}


\subsection{\TOM\ semantic}
%%%%%%%%%%%%%%%%%%%%%%%%%%%%%%%%%%%%%%%%%%%%%%%%%%%%%%%%%%%%

\subsubsection{Type definition}

\noindent
When defining a new type with the \lex{\%typeterm} constructs,
several access functions have to be defined:
\begin{itemize}
\item the \lex{implement} construct describes how the new type is 
  implemented. The target language part written between braces
  (\lex{'\{'} and \lex{'\}'}) is never parsed. It is used by
  the compiler to declare some functions and variables.

\item the \lex{get\_fun\_sym(t)} construct corresponds to a
  function (parametrised by a term variable) that should return the
  root symbol of a given term (the term referenced by the term
  variable \texttt{t} in this example). 

\item the \lex{cmp\_fun\_sym(s1,s2)} construct corresponds to a
  predicate (parametrised by two symbol variables).
  This predicate should return \texttt{true} if the symbols are
  ``equal''. The \texttt{true} value should correspond to the 
  builtin \texttt{true} value of the considered target language.
  (\texttt{true} in \Java, and something different from \texttt{0} in
  \Clang\ for example). 

\item the \lex{get\_subterm(t,n)} construct corresponds to a
  function (parametrised by a term variable and an integer).
  This function should return the \texttt{n-th} subterm of the
  term~\texttt{t}. This never called with and integer paramter that
  does not correspond to the arity of the root symbol of the
  considered term (i.e. we alway have $0 \leq n < arity$).

\item the \lex{equals(t1,t2)} construct corresponds to a
  predicate (parametrised by two term variables).
  This predicate should return \texttt{true} if the terms are
  ``equal''. The \texttt{true} value should correspond to the 
  builtin \texttt{true} value of the considered target language.
  This last optional predicate is only used to compile non-linear 
  left-handsides. It is not needed, if the specification does not
  contain such patterns.
\end{itemize}


\noindent
When defining a new type with the \lex{\%typelist} construct,
several other access functions have to be defined:
\begin{itemize}
\item the \lex{get\_head(l)} function is parametrised by a list
  variable and should return the first element of the considered list.

\item the \lex{get\_tail(l)} function is parametrised by a list
  variable and should return the tail of the considered list.

\item the \lex{is\_empty(l)} constructs corresponds to a
  predicate parametrised by a list variable.
  This predicate should return \texttt{true} if the considered list
  contains no element.
\end{itemize}

\noindent
When defining a new type with the \lex{\%typearray} construct,
the two different access functions have to be defined:
\begin{itemize}
\item the \lex{get\_element(l,n)} construct is parametrised by a list
  variable and an integer. This should correspond to a function that
  return the \texttt{n-th} element of the considered list~\texttt{l}.

\item the \lex{get\_size(l)} constructs corresponds to a function
  that returns the size of the considered list.
  By convention, an empty list contains \texttt{0} element.
\end{itemize}

\subsubsection{Operator definition}

\noindent
When defining a new symbol with the \lex{\%op} construct, the user
should specify how the symbol is implemented. This is done by the
\lex{fsym} construct.
The expression between braces should correspond (modulo the
\lex{cmp\_fun\_sym} predicate) to the expression returned by the
function \lex{get\_fun\_sym} applied to a term rooted by the
considered symbol.  
When defining a symbol, is it also possible to specify a
\lex{make} construct. This function is parametrised by several 
term variables (i.e. that should correspond to the arity of the
symbol). A call to this \lex{make} function should return a term
rooted by the considered symbol, where each subterm correspond to the
terms given in arguments to the function.

\noindent
As mentioned in the syntax definition, it is also possible to name
each field of a constructor symbol by using the
\texttt{Type f(name1:Type, name2:Type2)} syntax.  
Adopting this programming style has two main advantages:
\begin{itemize}
\item when writing a pattern, this allows you write \texttt{f[name2=a]}
  instead of \texttt{f(\_,a)}. One benefit is that you can modify the
  signature (adding a field for example) without necessary having to
  modify every pattern that occurs in the program.

\item you may also specialize the \lex{get\_subterm} access
function for a given constructor. This can be done with the
\lex{get\_slot} construct.
\end{itemize}

\smallskip\noindent
When defining a new symbol with the \lex{\%oplist} construct,
the user has to specify how the symbol is implemented. 
The user has also to specify how a list can be built (this is no
longer optional as in the \lex{\%op} construct):
\begin{itemize}
\item the \lex{make\_empty()} construct should return an empty
  list.

\item the \lex{make\_add(l,e)} construct corresponds to a function
  parametrised by a list variable and a term variable. This function
  should return a new list~\texttt{l'} where the element~\texttt{e}
  has been inserted at the head of the list~\texttt{l}
  (i.e. \lex{equals(get\_head(l'),e)} and
  \lex{equals(get\_tail(l'),l)} should be \texttt{true}).
\end{itemize}

\noindent
When defining a new symbol with the \lex{\%oparray} construct,
the user has to specify how the symbol is implemented. 
The user has also to specify how a list can be built (this is no 
longer optional as in the \lex{\%op} construct):
\begin{itemize}
\item the \lex{make\_empty(n)} construct should return a list of
  size~\texttt{n}.  

\item the \lex{make\_add(l,e,n)} construct corresponds to a
  function parametrised by a list variable, a term variable and an
  integer. This function should return a list~\texttt{l'} such that
  the element~\texttt{e} is at the \texttt{n-th} position.
\end{itemize}

\subsubsection{Match definition}

In order to match patterns against a list of subjects, \TOM\ provides
the \lex{\%match} construct.
This constructs contains two parts:
\begin{itemize}
\item a list of \nt{MatchArgument}: this is a list of (target
  language) variables that reference the terms to be matched.
\item a list of \nt{PatternAction}: this is a list of pairs
  (pattern,action), where an action is a set of target language
  instructions.  
\end{itemize}

The \lex{\%match} construct is evaluated in the following way:
\begin{itemize}
\item given a list of ground terms (referenced by the list of target
  language variables), the execution control is transfered to the
  first \nt{PatternAction} whose patterns match the list of ground
  terms.
\item given a \nt{PatternAction}, the list of free variables is
  instantiated and the associated semantic action is executed.
  If the execution control is transfered outside the
  \lex{\%match} instruction (by a \texttt{goto}, \texttt{break} or
  \texttt{return} for example), the matching process is finished.
  Otherwise, it is continued as follows:
  \begin{itemize}
  \item if the considered matching theory may return several matches
    (list-matching for instance), for each match, the list of free
    variables is instantiated and the associated semantic action is
    executed.
  \item when all matches have been computed (there is at most one match
    in the syntactic theory), the execution control is transfered to
    the next \nt{PatternAction} whose patterns match the list of
    ground terms. 
  \end{itemize}

\item when there is no more \nt{PatternAction} whose patterns
  match the list of ground terms, the \lex{\%match} instruction is
  finished, and the execution control is transfered to the next
  instruction. 
\end{itemize}

\noindent
\textbf{Note:} the behaviour is not determined if a semantic action
modifies a target language variable which is an argument of a
\lex{\%match} instruction under evaluation.

\subsubsection{Rule definition}

The \lex{\%rule} construct is composed of a list of rewrite rules
(the left-hand side is a term and the right-hand side is a term).
All these rules should begin with the same root symbol. The \TOM\
compiler should generate a function (with one argument) whose name
correspond to the name of this unique root symbol.
Given a ground term, applying this function returns the instanciated
right-hand side of the first rule whose pattern matches the considered 
subject.
When no rule can be applied (i.e. no pattern matches the subject),
the given ground term, rooted by the root symbol of the rewrite system
is returned.


%\newpage
%%%%%%%%%%%%%%%%%%%%%%%%%%%%%%%%%%%%%%%%%%%%%%%%%%%%%%%%%%%%
\part{Libraries} 
%%%%%%%%%%%%%%%%%%%%%%%%%%%%%%%%%%%%%%%%%%%%%%%%%%%%%%%%%%%%
\chapter{Runtime libraries}
\cutname{lib.html}
%%%%%%%%%%%%%%%%%%%%%%%%%%%%%%%%%%%%%%%%%%%%%%%%%%%%%%%%%%%%
\section{Runtime and traversal functions}

\section{Debug}


%\newpage
%%%%%%%%%%%%%%%%%%%%%%%%%%%%%%%%%%%%%%%%%%%%%%%%%%%%%%%%%%%%
\part{Tutorial}
%%%%%%%%%%%%%%%%%%%%%%%%%%%%%%%%%%%%%%%%%%%%%%%%%%%%%%%%%%%%
\chapter{Tutorial and examples}
\cutname{tutorial.html}
%%%%%%%%%%%%%%%%%%%%%%%%%%%%%%%%%%%%%%%%%%%%%%%%%%%%%%%%%%%%

On top of \TOM\ directory, you can find the tutorial directory where
all examples presnted here can be found and ran.
The difficulties encountered in this tutorial is increasing with each
section and subsection. Most of the examples are based on Java and the
corresponding ATerm library for terms representation.

\section{Peano integer}
The Peano integer formalism aims to represent integer as Zero and the
successor of an integer.
\subsection{Simple}
\TOM\ allows to define such algebraic specifications. First, we define
the sort term and the 2 defined operators: zero and suc.

\typeterm\ term \{\\
\begin{quote}
 \implement           \{ ATerm \}\\
 \getfs(t)      \{ (((ATermAppl)t).getAFun()) \}\\
 \cmpfs(t1,t2)  \{ t1 == t2 \}\\
 \gets(t, n)   \{ (((ATermAppl)t).getArgument(n)) \}\\
\end{quote}
\}\\
\\
\op\ term zero \{\\
\begin{quote}
\fsym \{ fzero \}\\
\end{quote}
\}\\
\\
\op\ term suc(term) \{\\
\begin{quote}
  \fsym \{ fsuc \}\\
\end{quote}
\}\\


\subsection{Advanced}

\section{Integer and Fibonacci}
\subsection{Simple}
\subsection{Advanced}

\section{List}

\section{Apigen and the automatic generation of term representation}

\section{Polynomial expression and \TOM}


\end{document}
